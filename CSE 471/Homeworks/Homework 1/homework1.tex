\documentclass[letterpaper, 12pt]{article}
\title{CSE471 - Homework 1}
\author{Kumal Patel}
\date{\today}


\begin{document}
    
\maketitle

\begin{enumerate}
    \item[Exercise 1.1] 
        A good real-world example to this problem would involve sports. The agent would need to play against 
        an opponent and be able to score more within a fixed amount of time. The performance measure is
        only concerned with the first T steps of the environment, therefore the duration of the game 
        would be the T steps. In a game of basketball the agent would interact with the basketball court, 
        their opponent, and the basketball. The agent would need to be able to move around and handle the 
        basketball which uses a variety of motors. It would also need cameras and other sensors to be able
        to sense the state of the environment which is used to determine the agents action. The agent 
        would also need a time sensor so the agent would know when the game is concluded.
    \item[Exercise 1.2] 
    The PEAS as discussed in class:
        P - clean the room
        E - dirt, location, vacuum
        A - left, right, suck, no-op
        S - location, status.
    \begin{enumerate}
        \item 
        False, this agent is rational but the task environment is only partially observable. This agent 
        location sensor is now local and cannot tell if there is dirt in the adjacent square. 
        With the PEAS remaining the same.
        \item 
        True, since a simple reflex agent performs actions based on its current percept, if your task
        environment is now changed to remember the order in which the dirt was sucked (A or B squares). 
        PEAS being: P - Recall the location of the dirt that has been sucked and clean the room,
        E - dirt, location, vacuum A - left, right, suck, no-op, display S - location, status. To recall the order
        of dirt that was being sucked the agent would need to remember the sequence of percepts, and not 
        just the current percept.
        \item 
        True, if the task environment equally benefits all rational agents. P - picks up dirt E - dirt, location, vacuum
        A - left, right, suck, no-op S - location, status. Since the task environment rewards the agent if it picks up dirt at least once
        then regardless of any other outcome the agent is rational.
        \item True, similar to the previous problem where all actions return the same reward, equally beneficial. 
        P - picks up dirt E - dirt, location, vacuum A - left, right, suck, no-op S - location, status.
        Selecting a random action that results uniformly will be rational, since the agent will still pick up dirt.
        \item False, an agent would need to perceive as it moves around to be rational. PEAS being: 
        P - clean the room E - n/a A - left, right, suck, no-op S - location, status. Since the environment is 
        unobservable the agent cannot detect anything in the environment; so the agent is therefore, not rational.
    \end{enumerate} 
    \item[Exercise 1.3] 
    \begin{enumerate}
        \item 
        No, since the agent can only perceive its current location and the status. This agent cannot tell
        weather there are obstacles around it, so if it crashes into an obstacle then the agent has no way of escaping.
        \item 
        If the simple reflex agent randomly moves until if finds dirt to suck this design would outperform
        the other simple reflex agent because the random movement helps to escape if it ever gets stuck 
        behind any obstacles.
        \item 
        Yes, if there are two large rooms connection via a narrow hallway, it will take a long time and more
        movements for the randomized agent to get to the other room. 
    \end{enumerate} 
    \item[Exercise 1.4]
    \begin{enumerate}
        \item A goal-based agent best describes this behavior. This agent has a goal in mind and makes a decision to 
        reduce the distance from its goal. In this scenario, the agent is trying to reach the goal, and the decision making is
        to make a left turn whenever it encounters an obstacle, then it proceeds to move towards the goal. 
        
        \begin{tabbing}
            if\=(agent is not in goal) \{ \\
             \> while\=(agent encounters obstacle) \\
                \> \> turn left; \\
                \> move towards goal; \\ 
            \}
        \end{tabbing}

        \item Yes, the agent will be able to find its way out from the paths exit to the goal state.
        The agent continuously makes a left when there is an obstacle in its path; otherwise, it will head 
        toward the goal if possible. So, with a closed obstacle that contains an exit the agent will head
        closest to the goal and if there is an obstacle there it will turn until it can freely move. So it
        will travel around until it finds the exit then head towards the goal.
        
    \end{enumerate}   
\end{enumerate}

\end{document}